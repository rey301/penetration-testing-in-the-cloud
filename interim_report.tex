\documentclass{report}
\usepackage[utf8]{inputenc}

\title{Penetration Testing On Different Cloud Based Services}
\author{Abraham Rey}

\begin{document}

\maketitle

\section{Introduction}
Cloud computing has become increasingly popular over recent several years and enterprises have become more reliant on these types of systems. It is a system with which you can access a shared pool of configurable computing resources over a network. Offering minimal management effort and/or service provider interaction [1] as the main reason why it makes it so appealing. 

Regarding penetration testing, it serves to search for vulnerabilities in a target and exploit them. Ultimately we achieve remote access and root (administrator) privileges in the system. 

Vulnerabilities in clouds are present due to many reasons. APIs, which are insecure, may have lack of input sanitisation and improper access control; servers are misconfigured, weak credentials, outdated software and even coding practices that are insecure [8]. All these may be exploited and lead to integrity and confidentiality breaches.

In this project we are exploring penetration testing on the popular Cloud Service Providers (CSPs) such as AWS, Azure, and GCP. We will be using Kali Linux which is an open-source security package that has tools divided by categories [12]. 

\section{Professional Considerations}
\section{Related Work}
\section{Requirements Analysis}
\section{Project Plan}
\section{Interim Log}
\section{Proposal Document}

\end{document}
